\documentclass[11pt]{article}\usepackage[]{graphicx}\usepackage[]{color}
% maxwidth is the original width if it is less than linewidth
% otherwise use linewidth (to make sure the graphics do not exceed the margin)
\makeatletter
\def\maxwidth{ %
  \ifdim\Gin@nat@width>\linewidth
    \linewidth
  \else
    \Gin@nat@width
  \fi
}
\makeatother

\definecolor{fgcolor}{rgb}{0.345, 0.345, 0.345}
\newcommand{\hlnum}[1]{\textcolor[rgb]{0.686,0.059,0.569}{#1}}%
\newcommand{\hlstr}[1]{\textcolor[rgb]{0.192,0.494,0.8}{#1}}%
\newcommand{\hlcom}[1]{\textcolor[rgb]{0.678,0.584,0.686}{\textit{#1}}}%
\newcommand{\hlopt}[1]{\textcolor[rgb]{0,0,0}{#1}}%
\newcommand{\hlstd}[1]{\textcolor[rgb]{0.345,0.345,0.345}{#1}}%
\newcommand{\hlkwa}[1]{\textcolor[rgb]{0.161,0.373,0.58}{\textbf{#1}}}%
\newcommand{\hlkwb}[1]{\textcolor[rgb]{0.69,0.353,0.396}{#1}}%
\newcommand{\hlkwc}[1]{\textcolor[rgb]{0.333,0.667,0.333}{#1}}%
\newcommand{\hlkwd}[1]{\textcolor[rgb]{0.737,0.353,0.396}{\textbf{#1}}}%
\let\hlipl\hlkwb

\usepackage{framed}
\makeatletter
\newenvironment{kframe}{%
 \def\at@end@of@kframe{}%
 \ifinner\ifhmode%
  \def\at@end@of@kframe{\end{minipage}}%
  \begin{minipage}{\columnwidth}%
 \fi\fi%
 \def\FrameCommand##1{\hskip\@totalleftmargin \hskip-\fboxsep
 \colorbox{shadecolor}{##1}\hskip-\fboxsep
     % There is no \\@totalrightmargin, so:
     \hskip-\linewidth \hskip-\@totalleftmargin \hskip\columnwidth}%
 \MakeFramed {\advance\hsize-\width
   \@totalleftmargin\z@ \linewidth\hsize
   \@setminipage}}%
 {\par\unskip\endMakeFramed%
 \at@end@of@kframe}
\makeatother

\definecolor{shadecolor}{rgb}{.97, .97, .97}
\definecolor{messagecolor}{rgb}{0, 0, 0}
\definecolor{warningcolor}{rgb}{1, 0, 1}
\definecolor{errorcolor}{rgb}{1, 0, 0}
\newenvironment{knitrout}{}{} % an empty environment to be redefined in TeX

\usepackage{alltt}
\usepackage[top=1.0in, bottom=1.0in, left=1 in, right=1 in]{geometry}
\renewcommand{\baselinestretch}{1.1}
\usepackage{graphicx}
\usepackage{natbib}
\usepackage{amsmath}
\bibliographystyle{..//refs/styles/nature.bst}
\def\labelitemi{--}
\parindent=12
\usepackage{fancyhdr}
 
\pagestyle{fancy}
\fancyhf{}
\rhead{Daniel Buonaiuto}
\lhead{DAC meeting: January 2021}
\IfFileExists{upquote.sty}{\usepackage{upquote}}{}
\begin{document}
\subsection{Chapter 1: Reconciling competing hypotheses regarding flower-leaf sequences in temperate forests for fundamental and global change biology}
Phenology is a major component of an organism's fitness. While individual phenological events affect fitness, growing evidence suggests that the relationship between events may be equally or more important. This may explain why temperate deciduous woody plants exhibit considerable variation in the order of reproductive and vegetative events, or flower-leaf sequences (FLSs). There is evidence to suggest that FLSs may be adaptive, with several competing hypotheses to explain their function. Here, we advance existing hypotheses with a new framework that accounts for quantitative FLS variation at multiple taxonomic scales using case studies from temperate forests. Our inquiry provides several major insights towards a better understanding of FLS variation. First, we show that support for FLS hypotheses is sensitive to how FLSs are defined, with quantitative definitions being the most useful for robust hypothesis testing. Second, we demonstrate that concurrent support for multiple hypotheses should be starting point for future FLS analyses. Finally, we highlight how adopting a quantitative, intra-specific approach generates new avenues for evaluating fitness consequences of FLS variation and provides cascading benefits to improving predictions of how climate change will alter FLSs and thereby re-shape plant communities and ecosystems.\\

\textbf{Status:} Published. Buonaiuto, D.M., Morales‐Castilla, I. and Wolkovich, E.M. (2021), Reconciling competing hypotheses regarding flower–leaf sequences in temperate forests for fundamental and global change biology. New Phytol, 229: 1206-1214. https://doi.org/10.1111/nph.16848

\subsection*{Chapter 2: Differences in flower and leaf bud environmental responses drive shifts in spring phenological sequences of temperate woody plants}
The relative timing of growth and reproduction is an important driver of plant fitness. For deciduous woody species in temperate regions leaves and flowers both appear in the early spring, but the order and duration of these phenological events vary among species, populations, and individuals. This observed variation in flower-leaf sequences (FLSs) suggests that the relative timing of these events may be important---affecting the reproduction, recruitment and survival of individuals. Further, FLSs appear to be shifting with climate change, and anticipating the extent of these shifts is key to projecting how climate change may impact species' performance and reshape forest communities. Predicting FLS shifts, however, requires an improved understanding of how the environment dictates FLS patterns. To address this, we compared the phenological responses of flower and leaf buds for 10 temperate woody species to varying levels of temperature and light in a lab experiment. Our experimental design allowed us to test competing hypotheses for how environmental cues determine FLS variation---specifically whether forcing (warm temperatures) alone drives variation or differential sensitivity to chilling (cool temperatures generally in the fall and winter) and/or photoperiod matter. We found that flower and leaf buds respond with differential sensitivity to environmental cues, with differences in their response to chilling being the dominant driver of FLS variation. Because climate change will amplify variability in chilling across time and space, these findings suggest that FLS shifts may be large, but are likely to vary substantially among populations and species. In our study, FLS shifts were largest in wind-pollinated species that flower before leafing, raising the possibility that, more generally, wind-pollinated taxa may experience reproductive declines due to FLS shifts in the decades to come.\\

\textbf{Status:} Submitted. Journal of Ecology, December 2020.

\subsection*{Chapter 3: Flower leaf sequences in the American plums}
\textbf{Project overview:} My previous two chapters demonstate the importance of FLSs in wind-pollinated taxa, yet these findings offer little information about the functional significance of hysteranthy in the many insect-pollinated species share this FLS. To fill in this gap, I have undertaken a new project that will explore the functional ecology of FLSs in insect-pollinated taxa by exploring associations between FLSs, species distributions and functional traits using a large data set of herbaria specimens.\\

In this project I am focuing on the American plums (\textit{Prunus} subsp. \textit{Prunus} sect. \textit{Prunocerasus}). The FLS patterns of these 16 (or 14) vary among species, yet all share a common flower morphology making then an ideal clade for trait comparisons. contains an additional 13,869. Leveraging the over 13,000 digitized herbarium specimens in The Consortium of Midwest Herbaria collections, I will a) quantify FLS differences among species, b) test the major hypotheses of the function of FLS variation by testing for correlations between FLS patterns and the functional traits and ecological requirements of these species and c) evaluate the impact of climate change of FLSs in this clade.\\

For b), I will test associations with:
\begin{enumerate}
\item petal length (insect visibility hypothesis)
\item palmer drought severity index (drought tolerance hypothesis)
\item low temperatures (drought to cold tolerance hypothesis)
\item fruit diameter/phenology (early flowering hypothesis)
\end{enumerate}

\textbf{Status:} On-going. Data collection and preliminary analyses for part a) and c) are complete. Data collection for part b) in on going with the help of UBC undergraduate Sophia Collins.

\textbf{Next steps:}
\begin{enumerate}
\item Complete flower petal measurements
\item Begin fruit diameter measurements
\item Prepare climate data
\item Joint model
\end{enumerate}

\textbf{Main questions for committee:} Do I need to test ALL of these hypotheses?

\subsection*{Chapter 4: Seasonal priority effects (might) alter inter-specific competition among herbaceous plants}

\textbf{Project Overview:}\\
\indent In a previous pilot study, I determined that variable climate conditions alter the germination rank of herbaceous species. In this project, I will evaluate how these changes in germination rank affect inter-specific competition in a pairwise competition experiment.
I planted seed of 3 herbaceous species under varying density arrangements, and assigned them randomly to two stratification treatments ( 6 and 10 weeks). In the 10 week treatment all species should germinate closer together, while the time lag amoung should be greater under the 6 week treatments. After a growing season (two harvest dates) I will harvest all plant material and biomass the species from each pot to determine how much these germination priorty effect impact the growth rate of each species. \\

\textbf{Status:} Ongoing. Germination observations concluded on 20 January 2021.

\textbf{Next steps}: Harvest 1: Febraray 2021- Dry and weigh. Harvest 2: March 2021 Dry and weigh.

\textbf{}

\subsection*{Side hustle(s)}
\begin{itemize}
\item Since entering the PhD program , I have worked with the Wolkovich lab on a meta-analysis about phenology in controlled-environment experiments. Two manuscripts have been published from this project with a third is in review. A forth project, focusing on the relationship between species' ranges and phenological cues is one of my major focuses.
\item Along with Cat Chamberlain, I have maintained observations in the Wolkovich lab common garden plot at Weld Hill. We plant to continue monitoring this garden for the duration of our PhDs, and are preparing to start analyzing the data from this experiment this semester.
\item I have analyzed my pilot dataset on seed germination of 12 woodland species and could decide to try and write a manuscript about it.
\item While no longer a part of my disertation my analyses comparing  photo and thermo. periodicity in experimental environments are complete and could be written up as well.
\item I am expecting twins at the end of April.
\end{itemize}

\subsubsection*{Teaching}
\textbf{SPU 22:} From the Big Bang to the Brontosaurus and Beyond (S18,S19)\\
\textbf{OEB 137:} Experimental Design and Statistics for Ecology (F20)\\
\textbf{OEB 55:} Ecology: Populations, Communities and Ecosystems (S20,S21)\\

\end{document}
