
The complex process of seed germination is regulated by the environment. Species differ significantly under which conditions the will germination. This regeneration niche is a critical filter in plant community ecology. As environmental conditions change with antrhopogenic cliamte change,it is the germination capacity of many species will be comprimised, shifting community dominance.

Seeds of many temperate species are physiologically dormant--require a period of exposure to cool temperatures to enable germination. As winter warming is happening more rapidly than summer, it might be this cold requirement is what limits some species to germination and stay in the communitiy. Whereas species with a weaker or more plastic cold requirement may win.

However we lack the ability to predict these dynamics in temperate systems. 1. While protocols for testing cold stratifications are well established, the aims of most studies are to characterize dormancy treatments. This means that the usually rely on a few cold treatments at extreme levels for statistical power.

Reanalyzing  soil  warming data, we found that stratification condidtions between Oct and April at a temperate forest on averaged only differeed by about 23 days but with high variation. Thus typical cold stratification treatments 0, 8 weeks 12 week might miss these dynamics.

But to think about the effects of shifting stratification regimes in the wild on plant communities a number of secondary questions must be ask.

1. What is the slope of the chilling effect.
2. How does stratification effect none physiological dormant species
3 How does stratification interact with other changing temperature
3. what is the fate of ungerminated seeds
4. What other germination parameters are affected?
5. Using experiments to parameterize