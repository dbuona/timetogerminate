\documentclass{article}\usepackage[]{graphicx}\usepackage[]{color}
% maxwidth is the original width if it is less than linewidth
% otherwise use linewidth (to make sure the graphics do not exceed the margin)
\makeatletter
\def\maxwidth{ %
  \ifdim\Gin@nat@width>\linewidth
    \linewidth
  \else
    \Gin@nat@width
  \fi
}
\makeatother

\definecolor{fgcolor}{rgb}{0.345, 0.345, 0.345}
\newcommand{\hlnum}[1]{\textcolor[rgb]{0.686,0.059,0.569}{#1}}%
\newcommand{\hlstr}[1]{\textcolor[rgb]{0.192,0.494,0.8}{#1}}%
\newcommand{\hlcom}[1]{\textcolor[rgb]{0.678,0.584,0.686}{\textit{#1}}}%
\newcommand{\hlopt}[1]{\textcolor[rgb]{0,0,0}{#1}}%
\newcommand{\hlstd}[1]{\textcolor[rgb]{0.345,0.345,0.345}{#1}}%
\newcommand{\hlkwa}[1]{\textcolor[rgb]{0.161,0.373,0.58}{\textbf{#1}}}%
\newcommand{\hlkwb}[1]{\textcolor[rgb]{0.69,0.353,0.396}{#1}}%
\newcommand{\hlkwc}[1]{\textcolor[rgb]{0.333,0.667,0.333}{#1}}%
\newcommand{\hlkwd}[1]{\textcolor[rgb]{0.737,0.353,0.396}{\textbf{#1}}}%
\let\hlipl\hlkwb

\usepackage{framed}
\makeatletter
\newenvironment{kframe}{%
 \def\at@end@of@kframe{}%
 \ifinner\ifhmode%
  \def\at@end@of@kframe{\end{minipage}}%
  \begin{minipage}{\columnwidth}%
 \fi\fi%
 \def\FrameCommand##1{\hskip\@totalleftmargin \hskip-\fboxsep
 \colorbox{shadecolor}{##1}\hskip-\fboxsep
     % There is no \\@totalrightmargin, so:
     \hskip-\linewidth \hskip-\@totalleftmargin \hskip\columnwidth}%
 \MakeFramed {\advance\hsize-\width
   \@totalleftmargin\z@ \linewidth\hsize
   \@setminipage}}%
 {\par\unskip\endMakeFramed%
 \at@end@of@kframe}
\makeatother

\definecolor{shadecolor}{rgb}{.97, .97, .97}
\definecolor{messagecolor}{rgb}{0, 0, 0}
\definecolor{warningcolor}{rgb}{1, 0, 1}
\definecolor{errorcolor}{rgb}{1, 0, 0}
\newenvironment{knitrout}{}{} % an empty environment to be redefined in TeX

\usepackage{alltt}
\IfFileExists{upquote.sty}{\usepackage{upquote}}{}
\begin{document}

\section{Introduction}
\subsection{Relative timing of germination, species interactions, and climate change}
\begin{enumerate}
\item The relative timing of germination is an important mechanism of species interactions; competition is often fiercest in early ontogony and chance of survival low.
\item Competitivive dominance and/or invasion success is often associated with rapid germination.
\item And this has been show to be mechaninsitic in experiments that staggered planting.
\item In most natural systems, the timing of germination is dicated by environmental cues rather than time of arival. Which means interannaul variation in envirnmental cues should correlate with interannual germination success.
\item If species have unique vital rate responses to their environment than 1) this can lead to coexistnace via the storage effect and 2) sustained alterations to environmental cues, ie, climate change has potetial to alter the relative timing of germination among species(hereafter: germination rank). 
\item Depending on the strength differences among species (priority effects), cliamte change could amplify these differences and permantly alter community level germination ranks, shifting the balance of species' interactions, impacting demography and community processes.
\end{enumerate}
\subsection{Evidence for these dynamics}
\begin{enumerate}
\item These dynamics have been show in annual desert and grassland systems where germination is primirly controlled by water availability. Give some examples.
However it is unclear whether these seasonal priorty effects on germination matter in temperate forest systems.
\end{enumerate}
\subsection{Why temperate forests may be different}
\begin{enumerate}
\item In temperate forest most species are perrenials (germination matters less for coexistance)
\item Physiological dormancy is common temperature (stratification and incubation) is the main driver of germination.
\item it is unclear the dfferences vital response rates to temperature are large enough to alter germination rank enough to impact priority effects/species interactions. 
\end{enumerate}
\subsection{Our study}
\begin{enumerate}
\item While assumptions of the storage effect need to be evaluated in temperate forests, we focus evaluating the extent to which species have unique vital rate responses to their spatial environment, and those differences may impact priorty effect. First, we performed a germination study to quantify unique vital rate responses to stratification and incubation temperatures for a suite of temperate herbaceous species to evlaute the potential for germination rank shifts with climate. The we extended to literture review of Young 2017 to evluate whether or not germination differences on this scale might influnce priority effect 
\end{enumerate}

\section*{Methods}
\subsection*{Species}
\begin{enumerate}
\item Mix of field and forest species because seed bank of forest is often old field
\item Mix of dormancy classes
\end{enumerate}
\subsection{Experimental Methods}
\subsection{Data anaylsis}
I think we're going with the survival model. I'll just mention survial models assume everything germinations, which is a bad assumption so we decided anything that the t50 was greater than 30 days (or other) did not germination higher than that.

\subsection{extension of literature review}
\begin{enumerate}
\item search terms, how many studies in Young 2017
\item how many we added
\end{enumerate}

\section*{Results}
\begin{enumerate}
\item table 1: Matrix of species differences under cliamte change and reguar conditions
\item figure 1: mu plots, shape
\item plot of germination ranks under each scenario
\item Supp table of lit reivew with quantification of reponses (x out of y studies found priority effects with germination dfferences of <7 days, 7-21 days etc)
\end{enumerate}


\section*{Discussion}
\begin{itemize}
\item Yes, it seem like the differences are big enough to alter germination rank
\item Next, we have to investigate if these differences drive performance differnces (priority effects)
\item Our study didn't include risks to early germination--stabalizing selection on germiantion time.
\item Germination may be less important in forest systems But germination may become more important as the need to migrate or distubrance regimes change.
\item Population differences, maternial effects etc not accounted for.
\item in forest germinations compete with ramets not just other seeds.
\item These reulst should fit into larger deomography models that include survial, reproductive output etc.
\end{itemize}


\end{document}
