\documentclass{article}\usepackage[]{graphicx}\usepackage[]{color}
%% maxwidth is the original width if it is less than linewidth
%% otherwise use linewidth (to make sure the graphics do not exceed the margin)
\makeatletter
\def\maxwidth{ %
  \ifdim\Gin@nat@width>\linewidth
    \linewidth
  \else
    \Gin@nat@width
  \fi
}
\makeatother

\definecolor{fgcolor}{rgb}{0.345, 0.345, 0.345}
\newcommand{\hlnum}[1]{\textcolor[rgb]{0.686,0.059,0.569}{#1}}%
\newcommand{\hlstr}[1]{\textcolor[rgb]{0.192,0.494,0.8}{#1}}%
\newcommand{\hlcom}[1]{\textcolor[rgb]{0.678,0.584,0.686}{\textit{#1}}}%
\newcommand{\hlopt}[1]{\textcolor[rgb]{0,0,0}{#1}}%
\newcommand{\hlstd}[1]{\textcolor[rgb]{0.345,0.345,0.345}{#1}}%
\newcommand{\hlkwa}[1]{\textcolor[rgb]{0.161,0.373,0.58}{\textbf{#1}}}%
\newcommand{\hlkwb}[1]{\textcolor[rgb]{0.69,0.353,0.396}{#1}}%
\newcommand{\hlkwc}[1]{\textcolor[rgb]{0.333,0.667,0.333}{#1}}%
\newcommand{\hlkwd}[1]{\textcolor[rgb]{0.737,0.353,0.396}{\textbf{#1}}}%
\let\hlipl\hlkwb

\usepackage{framed}
\makeatletter
\newenvironment{kframe}{%
 \def\at@end@of@kframe{}%
 \ifinner\ifhmode%
  \def\at@end@of@kframe{\end{minipage}}%
  \begin{minipage}{\columnwidth}%
 \fi\fi%
 \def\FrameCommand##1{\hskip\@totalleftmargin \hskip-\fboxsep
 \colorbox{shadecolor}{##1}\hskip-\fboxsep
     % There is no \\@totalrightmargin, so:
     \hskip-\linewidth \hskip-\@totalleftmargin \hskip\columnwidth}%
 \MakeFramed {\advance\hsize-\width
   \@totalleftmargin\z@ \linewidth\hsize
   \@setminipage}}%
 {\par\unskip\endMakeFramed%
 \at@end@of@kframe}
\makeatother

\definecolor{shadecolor}{rgb}{.97, .97, .97}
\definecolor{messagecolor}{rgb}{0, 0, 0}
\definecolor{warningcolor}{rgb}{1, 0, 1}
\definecolor{errorcolor}{rgb}{1, 0, 0}
\newenvironment{knitrout}{}{} % an empty environment to be redefined in TeX

\usepackage{alltt}
\usepackage{Sweave}
\usepackage{float}
\usepackage{graphicx}
\usepackage{tabularx}
\usepackage{siunitx}
\usepackage{mdframed}
\usepackage{natbib}
\bibliographystyle{..//refs/styles/besjournals.bst}
\usepackage[small]{caption}
\setkeys{Gin}{width=0.8\textwidth}
\setlength{\captionmargin}{30pt}
\setlength{\abovecaptionskip}{0pt}
\setlength{\belowcaptionskip}{10pt}
\topmargin -1.5cm        
\oddsidemargin -0.04cm   
\evensidemargin -0.04cm
\textwidth 16.59cm
\textheight 19.94cm 
%&\pagestyle{empty} %comment if want page numbers
\parskip 0pt
\renewcommand{\baselinestretch}{1.5}
\parindent 20pt

\newmdenv[
  topline=true,
  bottomline=true,
  skipabove=\topsep,
  skipbelow=\topsep
  ]{siderules}
\usepackage{lineno}
%\linenumbers

\usepackage{multirow}
\IfFileExists{upquote.sty}{\usepackage{upquote}}{}
\begin{document}

\begin{document}
\title{Temporal Biology Seminar: Phenology}
\date{}
\maketitle{}
\indent Phenology, the timing of annual life cycle events, is a critical trait in living organisms, influencing and being influenced by evolution and ecology across many scales. In this graduate seminar we will discuss the biological underpinning and implications of phenology through primary scientific literature. \\

\textbf{Course objects:} The goal of this course is for students to broadly consider the role that phenology plays in biological, ecological and evolutionary processes. Specfically, students will:
\begin{itemize}
\item Understand the biolgical and evolutionary underpinning of phenology.
\item Consider the role of phenology in structuring ecological and evolutionary processes.
\item Gain exposure to the diverse application of phenological research within the field of biology, and consider phenology's role in their own field of interest.
\end{itemize}

\textbf{Course Structure:} This seminar will meet once per week for two hours. Meetings will consist of brief introductory and concluding remakes by the instruction and a student facilitated discussion.\\

\textbf{Prerequisites:} Advanced academic standing with introductory coursework in ecology and evolution, or permission of instructor.

\begin{center}
\scalebox{0.8}{
  \begin{tabular}{|l|l|l|l|}
\hline
Week & Topic & Lecture Focus & Readings \\
\hline
\multicolumn{4}{|c|}{Part I: Fundamentals} \\
\hline
\multirow{2}{10em}{Week 1} &\multirow{2}{10em} {Introductory Remarks } &  What is phenology anyway?"& Forest and Rushing 2010 \\ 
& & and is phenology a "trait"? & Lechowicz 1984, Ollerton 1992, \\
\hline
\multirow{2}{10em}{Week 2} &\multirow{2}{10em} {Environmental Cues} & Environental factors influencing phenology & Rathke and Lacey 1985 \\ 
& & and modeling phenological responses to environments & Chuine 2000, Fu 2014  \\
\hline
\multirow{2}{10em}{Week 3} &\multirow{2}{10em} {Physiology of Phenology} & How/Where do plants perceive their environment? & Bernier 2005 \\ 
& & and genetic regulation of phenology & Wilczek 2010, Visser 2010  \\
\hline
\multirow{2}{10em}{Week 4} &\multirow{2}{10em} {Evolution of Phenology} & Heritability and local adaptation of phenology & Liepe 2016  \\ 
& & & McDonough-MacKenzie 2018, Vitasse 2009 \\
\hline
\multirow{2}{10em}{Week 5} &\multirow{2}{10em} {Carryover effects} & Phenological sequences & Gougherty 2018  \\ 
& & Maternal effects & Johnsen 2005, Auge 2017  \\
\hline
\multicolumn{4}{|c|}{Part II: Function} \\
\hline
\multirow{2}{10em}{Week 6} &\multirow{2}{10em} {Ecosystem Ecology} & Phenology, fluxes and feebacks  & Richardson 2013 \\ 
& &  & Fitzjarrald 2001, McKown 2012  \\
\hline
\multirow{2}{10em}{Week 7} &\multirow{2}{10em} {Community Ecology} & Temporal Niches & Fargione 2005 \\ 
& & Phenology and Competition & Ross 1972, Wainwright 2011  \\
\hline
\multirow{2}{10em}{Week 8} &\multirow{2}{10em} {Evolutionary biology} & Phenological speciation: Allochrony & Taylor 2017 \\ 
& & Phenology and life history evolution & Burghardt 2015, Rubin 2018  \\
\hline
\multicolumn{4}{|c|}{Part III: Phenology in a changing world} \\
\hline
\multirow{2}{10em}{Week 9} &\multirow{2}{10em} {Phenological Shifts} & Observed changes & Menzel 2006 \\ 
& &  & Ffrench-Constant 2016, Fu, 2015  \\
\hline
\multirow{2}{10em}{Week 10} &\multirow{2}{10em} {Invasion} & Phenology as a mechanism of invasion & Wolkovich 2013, Gioria 2017 \\ 
& & Rapid evolution of phenological response &  Franks 2007  \\
\hline
\multirow{2}{10em}{Week 11} &\multirow{2}{10em} {Phenology and Extremes} & False Spring & Gu 2008 \\ 
& & Drought & Ivits 2014, Cui 2017  \\
\hline
\multirow{2}{10em}{Week 12} &\multirow{2}{10em} {Phenological Mismatches} & Pollinator networks & Kudo 2003 \\ 
& &Herbivory and Predation  & Kharouba 2015, Petanidou 2014  \\
\hline

\end{tabular}}
\end{center}

\nocite{Auge_2017,Bernier_2005,Burghardt_2015,CHUINE_2000,Cui_2017,Fargione_2005,ffrench-Constant_2016, Forrest_2010,Franks_2007,Gioria_2016,Lechowicz_1984,Fu_2014,Fu_2015,Gougherty_2018,Gu_2008,Ivits_2013,Johnsen_2005,Kharouba_2015,Kudo_2013,Liepe_2016,McKown_2012,MENZEL_2006,OLLERTON_1992,PETANIDOU_2014,Rathcke_1985,RICHARDSON_2013,Ross_1972,Rubin_2018,Taylor_2017,Visser_2010,Vitasse_2009,Wainwright_2011,Wolkovich_2013,Wilczek_2010,Mackenzie_2018,Fitzjarrald_2001}
\bibliography{tempbio_ref}
\end{document}
