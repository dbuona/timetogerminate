\documentclass{article}\usepackage[]{graphicx}\usepackage[]{color}
%% maxwidth is the original width if it is less than linewidth
%% otherwise use linewidth (to make sure the graphics do not exceed the margin)
\makeatletter
\def\maxwidth{ %
  \ifdim\Gin@nat@width>\linewidth
    \linewidth
  \else
    \Gin@nat@width
  \fi
}
\makeatother

\definecolor{fgcolor}{rgb}{0.345, 0.345, 0.345}
\newcommand{\hlnum}[1]{\textcolor[rgb]{0.686,0.059,0.569}{#1}}%
\newcommand{\hlstr}[1]{\textcolor[rgb]{0.192,0.494,0.8}{#1}}%
\newcommand{\hlcom}[1]{\textcolor[rgb]{0.678,0.584,0.686}{\textit{#1}}}%
\newcommand{\hlopt}[1]{\textcolor[rgb]{0,0,0}{#1}}%
\newcommand{\hlstd}[1]{\textcolor[rgb]{0.345,0.345,0.345}{#1}}%
\newcommand{\hlkwa}[1]{\textcolor[rgb]{0.161,0.373,0.58}{\textbf{#1}}}%
\newcommand{\hlkwb}[1]{\textcolor[rgb]{0.69,0.353,0.396}{#1}}%
\newcommand{\hlkwc}[1]{\textcolor[rgb]{0.333,0.667,0.333}{#1}}%
\newcommand{\hlkwd}[1]{\textcolor[rgb]{0.737,0.353,0.396}{\textbf{#1}}}%
\let\hlipl\hlkwb

\usepackage{framed}
\makeatletter
\newenvironment{kframe}{%
 \def\at@end@of@kframe{}%
 \ifinner\ifhmode%
  \def\at@end@of@kframe{\end{minipage}}%
  \begin{minipage}{\columnwidth}%
 \fi\fi%
 \def\FrameCommand##1{\hskip\@totalleftmargin \hskip-\fboxsep
 \colorbox{shadecolor}{##1}\hskip-\fboxsep
     % There is no \\@totalrightmargin, so:
     \hskip-\linewidth \hskip-\@totalleftmargin \hskip\columnwidth}%
 \MakeFramed {\advance\hsize-\width
   \@totalleftmargin\z@ \linewidth\hsize
   \@setminipage}}%
 {\par\unskip\endMakeFramed%
 \at@end@of@kframe}
\makeatother

\definecolor{shadecolor}{rgb}{.97, .97, .97}
\definecolor{messagecolor}{rgb}{0, 0, 0}
\definecolor{warningcolor}{rgb}{1, 0, 1}
\definecolor{errorcolor}{rgb}{1, 0, 0}
\newenvironment{knitrout}{}{} % an empty environment to be redefined in TeX

\usepackage{alltt}
\IfFileExists{upquote.sty}{\usepackage{upquote}}{}
\begin{document}

\section*{Study objective}
The main objective of this study is to assess the effect of variable stratification periods on the rate of germination for a suite of temperate herbaceous species from forest edge/gap and open field environments.\\

\textbf{Main Questions:}
\begin{enumerate}
\item What is the relationship between stratification time and germination rate? Is there an optimum stratification periods above which germination is compromised?
\item More specifically, using the thermal time germination model,how do different stratification periods alter $t_b$ for the taxa in the study?
\item Does germination rank between species in a given community (forest edge and open) change due to variable stratification periods? What are the implications of these rank order changes for competition and community assembly in a changing climate?
\item Do the trends above differ between life histories, invasion status, evolutionary historiy or habitat type?
\end{enumerate}
\section*{Methods and Treatments}
\textbf{Overview:} Seeds of herbaceous  taxa are imbibed in distilled water for 24 hours and placed onto damp filter paper in petri dishes (20-40 seeds per dish).  Cohorts are sequentially removed from a cold stratification chamber and put into two growth chambera (lower temperature) and (higher temperature) for incubation. Germination fractions are recorded every 2 days.
\subsection*{Treatments}
\textbf{Cold stratification temperature:} All seeds will be stratified at 4 degrees C. This is within the range of reported optimum stratification temperatures (1-10 C, \citep{Bewley2013}) for a variety o taxa. This temperature has been used successfully by the lab for bud dormancy break on several occasions.\\
\textbf{Stratification duration:} For most taxa in the study, there are enough seeds for 10 different stratification periods. The reccomended stratification periods of the taxa range from 28-84 days, with the majority requiring between 42 and 56 days. In order to parameterize the change in $t_b$ in the thermal time model and assess whether there is in optimum stratification threshhold, I will use periods well below and above the optimum. Suequential removal from stratification ever 10 days allows for the coverage of a large stratification window while still providing enough resolution to plot a fine scale germination time course for each species. The stratification periods will be 0, 10, 20, 30, 40,50,60,70,80, 110 days (to investigate an upper limit).\\
\textbf{Incubation conditions:} Our goal is to simulate realistic germination conditions for taxa germination in the temperate Northeast. My treatments simulate May growing conditions in western Massachusetts (Harvard Forest) with day/night temperature fluctuations and 14 hour photo period. To understand the interaction between stratification and incubation temperate I chose a higher temperature secenario (reflecting historical extremes) and a lower temperature scenario (reflecting historical averages).\\
The 75 percent quantile of max May temperature at Harvard forest is 23. The 50 percent is 17.\\
The 25 percent quantile of daily temperature fluctions at Harvard forest is 11 degrees.\\
I will use 25/15 and 20/10. Matching the harvard forest fluctions and germination trial conventions (see Meyer).\\

\textbf{Incubation period:} As per Baskin and Baskin, germination trial (incubation period) will not exceed 25 days.\\

\section*{Predictions}
\begin{enumerate}
\item Many species will show no germination at no, or low stratification periods during the germination periods.
\item Increasing stratification periods will increase germination rate (shorter time to germination) for any species with physiological dormancy. For non-dormant species, stratification will not affect germination rate. Increasing stratification period will also allow  PD species to germinate more readily under the cooler temperature regime.
\item Excess stratification will not have detrimental effects on germination, but at some level or stratification, germination rate will plateau.
\item Stratification will reduce $t_b$ for all species
\item the strength of the responses predicted above with vary by species.
\item Germination rank will change between treatments.
\item Generally, invasive, annual and open field species will have faster germination rates.


\end{document}
