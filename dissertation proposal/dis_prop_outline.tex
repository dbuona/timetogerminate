\documentclass{article}\usepackage[]{graphicx}\usepackage[]{color}
%% maxwidth is the original width if it is less than linewidth
%% otherwise use linewidth (to make sure the graphics do not exceed the margin)
\makeatletter
\def\maxwidth{ %
  \ifdim\Gin@nat@width>\linewidth
    \linewidth
  \else
    \Gin@nat@width
  \fi
}
\makeatother

\definecolor{fgcolor}{rgb}{0.345, 0.345, 0.345}
\newcommand{\hlnum}[1]{\textcolor[rgb]{0.686,0.059,0.569}{#1}}%
\newcommand{\hlstr}[1]{\textcolor[rgb]{0.192,0.494,0.8}{#1}}%
\newcommand{\hlcom}[1]{\textcolor[rgb]{0.678,0.584,0.686}{\textit{#1}}}%
\newcommand{\hlopt}[1]{\textcolor[rgb]{0,0,0}{#1}}%
\newcommand{\hlstd}[1]{\textcolor[rgb]{0.345,0.345,0.345}{#1}}%
\newcommand{\hlkwa}[1]{\textcolor[rgb]{0.161,0.373,0.58}{\textbf{#1}}}%
\newcommand{\hlkwb}[1]{\textcolor[rgb]{0.69,0.353,0.396}{#1}}%
\newcommand{\hlkwc}[1]{\textcolor[rgb]{0.333,0.667,0.333}{#1}}%
\newcommand{\hlkwd}[1]{\textcolor[rgb]{0.737,0.353,0.396}{\textbf{#1}}}%
\let\hlipl\hlkwb

\usepackage{framed}
\makeatletter
\newenvironment{kframe}{%
 \def\at@end@of@kframe{}%
 \ifinner\ifhmode%
  \def\at@end@of@kframe{\end{minipage}}%
  \begin{minipage}{\columnwidth}%
 \fi\fi%
 \def\FrameCommand##1{\hskip\@totalleftmargin \hskip-\fboxsep
 \colorbox{shadecolor}{##1}\hskip-\fboxsep
     % There is no \\@totalrightmargin, so:
     \hskip-\linewidth \hskip-\@totalleftmargin \hskip\columnwidth}%
 \MakeFramed {\advance\hsize-\width
   \@totalleftmargin\z@ \linewidth\hsize
   \@setminipage}}%
 {\par\unskip\endMakeFramed%
 \at@end@of@kframe}
\makeatother

\definecolor{shadecolor}{rgb}{.97, .97, .97}
\definecolor{messagecolor}{rgb}{0, 0, 0}
\definecolor{warningcolor}{rgb}{1, 0, 1}
\definecolor{errorcolor}{rgb}{1, 0, 0}
\newenvironment{knitrout}{}{} % an empty environment to be redefined in TeX

\usepackage{alltt}
\usepackage{float}
\usepackage{graphicx}
\usepackage{tabularx}
\usepackage{siunitx}
\usepackage{mdframed}
\usepackage{cite}
\usepackage{natbib}
\bibliographystyle{..//refs/styles/besjournals.bst}
\usepackage[small]{caption}
\setkeys{Gin}{width=0.8\textwidth}
\setlength{\captionmargin}{30pt}
\setlength{\abovecaptionskip}{0pt}
\setlength{\belowcaptionskip}{10pt}
\topmargin -1.5cm        
\oddsidemargin -0.04cm   
\evensidemargin -0.04cm
\textwidth 16.59cm
\textheight 21.94cm 
%&\pagestyle{empty} %comment if want page numbers
\parskip 0pt
\renewcommand{\baselinestretch}{1.1}
\parindent 20pt

\newmdenv[
  topline=true,
  bottomline=true,
  skipabove=\topsep,
  skipbelow=\topsep
  ]{siderules}
%\usepackage{lineno}
%\linenumbers
\IfFileExists{upquote.sty}{\usepackage{upquote}}{}
\begin{document}
\title{A race against time: phenological sensativity as a mediator of plant interactions}
\author{Dissertation proposal of Daniel Buonaiuto}
\maketitle{}
\section*{Motivation and Dissertation Framework}
\par Phenology, the timing of annual life cycle events \citep{}, allows organisms to synchronize important life cycles events with optimum environmental conditions \citep{}. Phenology is an important mediator of ecosystem processes \citep{},species interactions \citep*{} and life history evolution \citep{}, and plays a major role in determining species' range limits \citep{}. Pronounced shifts in phenology observed across a broad range of taxa are also on of the most visable effects of antropogenic climate change to date \citep{}. On average, plant phenology has shifted by 4-8 days per decade, and it remains unclear whether these shifts will persist as seasonal climatic patterns shift global temperatures continue to rise. Phenological shifts of the magnitude we are already seeing are expected radically alter ecosystem structure and functioning on virtually all scales \citep*{}, however, the ways in which plant phenology interacts with ecosystem proccess and community function are complex, and ability to predict the precise nature of these changes remains limited. Addressing anthropogenic global change is one of the most critical existential challenges of our world today \citep{}, and it is essential that biological research is aimed to better understand the effects of global change on our planet.\\
\par In my proposed disseratation, I heed this call to advance our understanding of the effects of global change on species interactions as mediated through phenology. Sitting at the nexus of traditional community ecology and global change biology, my proposed disseration will explore how small differences in sensativity to environmental conditions can express significantly differnt phenological patterns, which in turn may alter inter and intra specific interactions in a dramatic way.\\
\par In \textbf{Chapter I}, I will examine flower-leaf sequences in temperate woody plant species, using experimental manipulations and statistical modeling to evaluate the currently evolutionary hypotheses for this phenological trait, characterize the natural range of variability in these patterns, and identify a possible mechanism for this variablity.\\
\par In \textbf{Chapter II}, I turn my attention to seed germination phenology, where I use probe the effect of variable stratification periods and incubation temperature on germination of a large suite of temperate, herbacious plants, to better understand species responses changing climate and the interaction between winter and spring temperature regimes.\\
\par In \textbf{Chapter III}, I expand the scope of my inquiry in chapter II, compiling a global database of experimentally manipulated germination trial, and using meta-anaylsis techniques to better understand the reaction norm of germination phenology in response to environmental manipulations.\\
\par In \textbf{Chapter IV}, I attempt to produce a proof of concept for the seasonal priority effect hypotheses, which predicts that species with ealier or more rapid phenological initiation will gain a competive advantage over their neighbors, by using alternate climatic treament to manipulate the germination phenologies or two species growing in a pairwise competition experiment.\\

\section*{Background}
 In the temperate zone, plant biolgical activity, growth and reproduction, occurs during the warmth of summer, while in the unfavorable conditions of winter, most plants enter a state of dormancy in which biological functioning is minimal. This phenological period of transiton between winter and spring, the release from dormancy and initiation of growth and reproduction is critical for plant function and reproduction, and is the focus of my work. Plants sense and interpret a complex set of environmental cues that signal the changing of the seasons, and can respond by fine-tuning their pheological transitions accordingly. For temperate plants, it is widely accepted that phenological transitions are responses to the interaction of exogenous environmental conditions like temperture and photoperiod with endogenous cues like circadian clocks. The cues seem to be conserved across ontogeny, as temperature and light are considered to be the main drivers or seed germination \citep{}, though the response to cues may vary over an individual's life time \citep{}. Below, I will briefly review: 1] bud and seed dormancy, 2] model of seed dormancy and germination 3] the development of the priority effect paradigm and the evidence that it opporates seasonally.\\
\ident\textbf{Dormancy, Defined and Explained:} Dormancy is as a tempory state in which metabolic activity is mimimized, preventing organism growth, development or activity. It is generally accepted that dormancy allows organisms to persist by conserving energy during periods that are unfavorable for biological functioning\citep{}. Bud dormancy  is generally divided into two stages endodormancy, where biological activity is surpressed by a build up of callose, isolating cells and preventing their activity even if growth conditions were favorable. Exposure to cold break down this callose, and the bud transitions from endo to ecodormancy, a state where biological activity can be reinitiate  after a certain exposure to favorable conditions warm temperature and increasing photoperiods.\\
The taxonomy of dormancy in seed is far more complicated than in buds. In their widely accepted taxonomy of seed dormancy, Baskin and Baskin \citeyear{} have indetified 5 major dormancy classes which can be subdivided into many more subclasses. Physiological dormancy (PD) is by far the most common dormancy class, especially in the temperate zone \citep{}. Seeds with PD are water permiable, but have a physiological inhibition mechanism in the embryo that prevents radical emergence \citep{}. Abscisic acid (ABA), a hormone upregulated during seed maturation is gernerally thought to inintiate and maintain dormancy as its concentation tends to decrease as seeds approach germination\ citep{} though the actual dirrect role of ABA in dormancy induction remains unclear. Environmental treatments such as cold or warm stratifiaction and afterrippening, have been shown to contribute to the degratation of ABA, and an upregualtion and increased sensativity to the growth hormone gibberelic acid ($GA_3$) which promotes germination \citep{}. It should be not that there is a signifcant debate in seed science research as to whether dormancy and germination are distinct phenophase (akin to endo and eco dormancy in buds) or whther germination is better considered as the end of the drmancy phase \citep{}.
\indent The optimum timing for the transition from dormancy to growth for buds and seeds are probably under similar selection pressures. Early initation is favored to extend the growing season as much as possible and avoid preditors, but not so early that to expose the vulnerable plant tissues to the perils of late season frosts. For flower buds, the availability of pollinators may be an additional selection factor. Many models have been developed to predict phenology in trees, while seed science most commonly uses a single model varient, the thermaltime or hydrothermal time model. As this model will be critical for Chapter II and III of my proposed disseration, I will present the model in more detail below.\\
\indent\textbf{The thermaltime model and its dormancy varient:} The thermaltime model of germination describe the relationship between time, temperature and germination fraction. The hydrothermaltime varient incorperates parementers for water potential($theta$) in model, but because my dissertation is focused on spring germination in the temperate zone where temperture rather than water potential would be expected to limiting, I will focus on the thermaltime only varient:
[insert model here]
Several attempt have been made to incorperate dormancy into this model. Conceptually, dormancy break treatment work to reduce $T_b$ or($theta_b$ in the hydrothermaltime varient) allowing for a more rapid accumulation of thermaltime and more rapid germination. This framework has been applied to include afterrippening \citep{}, and cold stratification\citep{}, but has not been broady applied outside of important agricultral or horticultural species, and should be applied broadly accross a diversity of taxa \citep{}. The model formulation is below:
[insert second model here]\\

\testbf{Priority Effects}: The lottery model of Chesson and Warner \citeyear{} is a core tenant of community assembly theory, and proports if many organisms are competing for a resource that the chance arrival of one of them will result in it success. This model evolved into the a more general catagory of priority effects, the impacts that a given species has on a system due to its prior arivival which can lead system on radically diverging tradjecteries to alternate stable states. [reread fukami] before this
\begin{itemize}
\item Classification of priority effects
\item that they've been demostrate
\item but they exist temporally
\item evidence
\

\section*{chapter I: The signifcance of flower-leaf sequences in an era of global climate change}
\subsection*{Introduction}
Why do some tree species seasonally flower before leafing out? This sequence, known as hysteranthy, proteranthy, or precocious flowering is readily apparent in many ecologically and commercially important species and has been described as  the characteristic flower-leaf sequence (FLS) of temperate deciduous forests \citep{}. Questions regarding the evolutionary origins and functionality of FLS are go beyond simple observational curiosity. Recent work has begun to show that it is not only individual phenological phases that are important, but also the realtionship between them, even if they are seemingly disperate \citep{}. In fact, most of the current hypotheses regarding FLS's suggest they critical to the reproductive or physiological functioning of woody plants \citep{}. Several authors suggest that the hysteranthous FLS is a trait critical for wind-pollination effeciency \citep{}. Others suggest that flowering first is an adaptation to reduce water stress and maintain floral hydration \citep{}, though this hypothesis has emerged primarily from the dry-decidous tropics where hysteranthy is also common \citep{}.  Still others suggest the hysteranthous FLS is and adapation to allow for extremely early flowering and is correlated with other early flowering traits such as seed size, dispersal time and cold tolerance \citep{}. It is also possible that FLS's are highly conserved trait, and the preponderance of hysteranthy in the temperate zone is a product of phylogenetic representation of the region rather than an adaptive quality to the trait.\\
\indent Despite the rich theoretical attention FLS has recieved in the literature, data about FLS is limited. The most comprehensive source of data we have regarding FLS comes from qualative descriptions in regional flora and guide books. A few long term empirical dataset can be found, but these are rare when compared with the larger body of phenological data as flower and leaf phenology have generally been observed separately{}. In part Im I ask, given the available data:
\begin{enumerate}
\item  What is the associations between FLS and several other life history traits (pollination syndrome, shade tolerance, plant height, flowering time, duration of fruit maturation) pertinent to the established hypotheses. Are these results sensative to data quality, observational ambiguity and modeling choices?
\item To what degree do FLS vary interantually, between populations, and individals?
\item Is there evidence that patterns FLS are being affected by global climate change.
 \end{enumerate}
Then, using a controlled environemnt experiment, I will investigate a possible mechansim that would produce variability in FLS over time and space, and extrapolate my findings to consider the effect of global cliamte change. In this part II I ask:
\begin{enumerate}
\item Is variablility in FLS a product of a differntial sensativies to environmental cue combinations between flower and leaf buds?
\end{itemize}

\subsection*{Proposed methods:}
\subsection*{Status and Preliminary Results:}
\textbf{What is the associations between FLS and several other life history traits?  Are sensativity of the above analysis to data quality, observational ambiguity and modeling choices?}
In the catagorical models, only flowering time is a consistantly strong predictor of FLS, with the effect size of pollination syndrome, seed development time and phylogenetic signal varing in significance depending on data source and modeling choices. There was no strong signal from height or shade tolerance. [See figure].\\
In the continuous model, pollination syndrome was the strongest predictor of FLS with shade tolerance and early flowering significant as well. [See figure]. These finding support the pollination syndrome and early flowering.\\
\textbf{To what degree do FLS vary interantually, between populations, and individals?}
There is signifcant variation between interanually and between indivudals. [See figure]. There are population difference, but these do not correlate to altitude or latitude, except in super hysteranthous species Alnus and Corylus.\\


\section*{Chapter II: The germination response to varying stratification regimes of a suite of temperate herbaceous speicies}
\subsection{Introduction}
Non-deep physiological dormancy the most common seed dormancy class in species from the temperate zone \citep{Finch-Savage_2006}. For many of such species, dormancy release is acheived by exposure to a period of cool, moist conditions prior to encountering suitable germination conditions \citep{}. This cold stratification requirement has been indetified in a large number of taxa, and stratifcation treatments are employed widely in both plant science and industry \citep{Hartmann_2011}. The temperatures under which stratification is acheived vary considerably among species, and has been reported to be quite broad in some \citep{Pritchard_1999,Vleeshouwers_2001}, and narrow in others \citep{WASHITAN_1988}. The duration of stratification needs has also been show to differ between species, ranging from just a few days \citpe*{} to many months \citep{} and may vary signifcantly between seed crops due to maternal effects  \citep{}. While cold stratifcation is commonly found as an expermental treatment in the literature, and has been shown to advance germination, studies which evaluate the germination response across a range of stratifaction periods or temperature are comparately more rare, and have typically been restricted important horticultural or agricultual species \citep{}. As a result, the dynamism of the germination response to variable statification regimes is poorly characterized in the vast majority of plant species.\\
\indent Cold stratification in the lab, serves as a proxy for natural exposure to chilling conditions a seed would experience overwintering in the field. With global climate change, changes to the severity and duration of winter will alter the natural stratification period experienced by seeds \citep{}. While winters are gernally predicted to be warmer and shorter \citep{IPCC}, the number of days which the stratification conditions are met may increase, decrease, or shift temporally, differentially affecting the germination phenology plant species depening on their geographic position, and the dynamics of their response to temperature \citep{Walck_2011}. These shifts in germination may in turn alter plant competition through priority effects \citep{} and plant demography though seed bank dynamics, and multi-trophic interactions.
\indent To better predict the effect of warming winters on seed germination, it is imperative to better characterize the germination response to variable stratification regimes for a more broad range of plant species. In this chapter I ask:
\begin{enumerate}
\item How does varying stratifaction periods effect the germination time courses of plant species?
\item How do stratification periods and incubation temperatures interact in germination time course?
\item Is the stratification requirement best characterized by an optimum (Can seeds get over stratified) or a threshold?
\item To what degree is germination rank between species affected by varying stratification regimes?
\end{enumerate}
\subsection*{Proposed methods}
\textbf{Experimental Protocols:} In the summer of 2018, seed of 24 temperate Eastern North America herbaceous plant species of both native and non-native origins were procurred from plant nursury stock or collections (see figure X for species and provanances) and dry-stored until the start of the experiment. In mid-August 2018, all seeds were checked for the prescence of an embryo using a float test \citep{Baskin_2014}, and imbibed in distilled water for 24 hours. Seeds of each species were then randomly divided into cohorts of 20-40, depending on seed availability and place onto wetted filter paper in 8 cm plastic petri dishes. Each cohort was then assigned to a combination of stratification duration (10 levels: 0, 10, 20, 30, 40, 50, 60, 70, 80,100 days) and incubation (2 levels, low temperature: 20 day/10 night or high 25 day/15 night) treatments, making for a 20 level fully factorial treatment design. For stratification, petri dishes were wrapped in aluminum foil and place in a X germination chamber in the dark at 4 degrees C. At the end of each stratification duration, cohorts were transfered to incubation conditions in growth chambers. Germination fractions were observed every other day for 25 days, and petri dishes were checked daily and moistened as needed. By measuring the germination fraction over time, I will generate germination time courses for each species and each treatment.\\
\textbf{Statistical analysis:} Using the germinination time courses for each species, I will calculate the rate of change for T_b given as a function of stratification duration. From these data, I will assess each species sensitivity to stratification, and use this information to predict how germination rank may change under different stratification scenarios. I will also examine whether species with different traits such as life history, habitat requirements, native status, and phylogeny differ in the strength of their response to the environmental treatments.\\
\subsection{Status/preliminary results}
\textbf{Project Status:} The experimental procedures are underway, and expected to conclude in December 2018.\\
\section*{OEGRES: A meta-analysis of Observed environmental germination resposes in experimental settings}
\subsection*{Introduction and Questions:}
\indent Seed germination is a critical life history stage for plant life, and as such, there is a long history, dating back to 300 years before the common era, of germination research \citep{Baskin_2014}. This work has produced a large body of literature detailing the germination requirements and dormancy classes for a vast number of taxa across an array of taxonimic and geographic space. Many comprehensive books and review papers have been written on th subject \citep*{},and germination reseach continues at a rapid pace around the globe, but there is still large gaps in our knowledge \citep{Baskin_2014}. As mentioned above, our understanding of  dynamics of germination response to environmental state remain in its infancy. Temperature is considerd to be the dominat factor which controls germination phenology,\citep{}, and as such, climate change is expected to have a strong impact on plant regeneration from seed. Without a better understand of a more complete range of germination responses to different environmental conditions, it is difficult to predict the extent of these climate change impacts. While few individuals studies sytematically investigate responses to a wide range of environmental conditions, in addition to the experimental work I proposed in chapter 2, the large body of germination literature could be leveraged to this end. In this chapter, I propose a meta-analysis to more broadly address the questions I laid out in chapter II:
\begin{enumerate}
\item How does varying environmental treatments effect the germination time courses of plant species?
\item How do various envrionmental treatment interaction to effect seed germination?
\item Do we detect any broad pattern in the germination response to variable environmental conditions on a phylogentetic, geopgraphic, or life-history level?
\end{enumerate}
\indent In addition to these fundamental biological question I will also use this project to address several important questions about germination research methodologies including:
\begin{enumerate}
\item Which environmental cues are most commonly manipulated?
\item What treatment levels most commonly utalized in research, how often are multiple treatment levels applied within one experiment?
\item Do treatments decisions correlate with other factors such as geography, institution, or taxa studied?
\end{enumerate}
\subsection*{Proposed methods:}
\indent Due to the acknowledged importance of temperature in mediating seed dormancy and germination, my primary interest is to evaluate the effects of stratifaction and incubation treatments on germination, but I intend to also include other environmental factors, such as water status, soil properties, afterripening, and scarification treatment in my analysis. The details proposed my data collection, study inclusion criteria and analysis are found below.\\
\textbf{Data collection:} I performed search of the Web of Science datebase using the key words "germination" and "stratification", and excluding meeting abstracts, abstracts of published items and proceedings papers. The search return 1,200 papers. I then read the abstract, methods and figures of these papers to identify studies that fit for inclusion in my analysis.\\
\textbf{Inclusion criteria:} To be included in the study, papers were required to:
\begin{itemize}
\item Report an temporal germination response in addition to a final germination percentage.
\item Manipulate a temperature variable (cold or warm stratification).
\end{itemize}
\indent Papers that were not fully accessible were also excluded from my analysis. Because of the large volume of papers, I will randomy sample 2000 papers that meet the selection criteria, and use them to build a database using ImageJ to scrape data from the relevant figures. The response variables I will capture include any measurement or index of germination rate as well as final germination percentages. The predictors I capture in addition to the specific environmental treatments in the paper will include:
\begin{itemize}
\item seed provenance (continent, latitude, longitude, altitude, biome)
\item seed age
\item year of collection and maternal environmental
\item dormancy class (if applicable)
\item non-environmental treatments (chemical application)
\item life history
\item population native status
\end{itemize}
\textbf{Analysis:} Upon completion of the database, I will use a multilevel, Baysian modeling framework to assess the impact of varying temperature regimes on germination behavior. I intend to run several models with differnt grouping factors, including taxonomic, regional. I will also query to database to address the methodological questions addressed above.
\subsection*{Project status:}
As of August 16, I have evaluted 610 papers of which 241 were determined fit for inclusion. Assuming this 39.5 percent inclusion rate remains consistant, I expect that 450-500 studies will be fit for inclsion.
\section*{Chapter IV: Seasonal priority effects: Germination phenology as a mediator of plant competition}
\subsection*{Introduction}
\subsection*{Proposed Methods}
\indent Based on the results of from chapter II, two species will be selected for a pairwise competition experiment. The two species will be selected based on the following criteria:
\begin{enumerate}
\item They have similar growth requirements and would be likely to be found in the same habitat in nature.
\item Their germination rank changes or the lag between their respective germination phenology shifts by more than 10-15 days given different stratification/incubation combinations.
\end{enumerate}
Seeds of each species will be sown in a soil medium at varying relative and overall densities following a response surface design detailed in \cite{Inouye_2000} [see figure]. This design has been shown to be the most effective for differentiating between the effects of intra vs. interspecific competion and integrating data with theoretical models of competition.  Replicants of each density will be randomly assigned to two different stratification/incubations regimes that have previously been show to alter the germination rank of the species, thus manipulating the strength of the seasonal priority effects in the competitive system.\\
After the given stratification period and 25 days of incubation, all pots will be transfer to greenhouse for the duration of the experiment (16 weeks). Every 4 weeks, the height of each plant will be measured and standardized photos of each pot will be taken to allow for an estimation of percent cover of each species. The measurement will be applied for a biomass estiamation using models found in \cite*{Axmanova_2012}. At each measurement interval, 5 plants of each species, not included in the response surface will be measured, thqn harvested, dried and weighed to better calibrate the biomass models. At the conclusion of the experiment, all plant will be harvested, dried and weight for a final biomass calculation.\\
\textbf{Analysis:} I will use the repeat measures of biomass to calculate and compare the relative growth rate (RGR) \citep{} of each species under the different priority effect manipulations. I predict that first species to germinate in each treatment will have a higher relative growth rate and surpress the growth rate of the second species. If no switch in germination rank is possible, I expect more pronounced priority effect (great lag between germination), to produce a greater differntial in relative growth rate than the weaker priority effect treatment.\\  


\section*{Timeline}
\begin{center}
\begin{tabular}{|l|l|}
Time & Task\\
\hline
\end{tabular}
\end{center}
\end{document}
